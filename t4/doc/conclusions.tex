\section{Conclusion}
We were expecting, from the OP, to get similar results between the two analysis: the voltages can be compared by: $node 2\ \ gain\equiv base$, $node 3\ \ gain\equiv node 1 \ \ output\equiv coll$, $node 4\ \ gain\equiv node3\ \ output \equiv vcc$,$node5 \ \ gain\equiv emit$, $node 2 output\equiv emit2$. The most dissimilar is $node 2 output\equiv emit2$.
The results differ, however, a bit. The differences are due to the existence of non-linear components, like capacitors and transistors. \\
In the incremental analysis, we notice that the input impedance in the circuit from \textit{ngspice}, independent of frequency, is close to the input impedance when $R_{E_1}=0$. It is not equal to any of the ones calculated by octave, since in \textit{ngspice} the capacitor is working, but in octave we consider it a short of an open circuit.\\
The output impedance in the two analysis, was considerably different, and the simulation result was much closer that of the $8 \Omega$ load. The smallest result was from the theoretical analysis, which is closer to the ideal $0 \Omega$. This may be due to all the approximations close to ideal made in this analysis, and not present in the simulation, which is closer to the real case.\\
Comparing the total gains from both analysis, we can see they are quite close, with \textit{ngspice} having its maximum at a gain of around 40db and \textit{octave} having its at around 44db. 
