\section{Results}
\label{sec:res}
\subsection{Output Voltage Gain}
The output voltage gain at the central frequency of 1000Hz, from the theoretical analysis has 35.657 dB magnitude, with a phase of 0.021517 rad.
\subsection{Frequency Response}
The plots of the output voltage gain as a function of frequency are shown below.
\begin{figure}[H] \centering
\includegraphics[width=0.4\linewidth]{vo2f.pdf}
\caption{Voltage gain in decibels as a fuction of Frequency (Simulation)}
\label{fig:vos}
\end{figure}

\begin{figure}[H] \centering
\includegraphics[width=0.4\linewidth]{gain.eps}
\caption{Voltage gain in decibels as a fuction of Frequency (Theoretical Anlysis)}
\label{fig:vot}
\end{figure}
As we can see, the plots are very similar.\\
\\
The plots of the phase of the gain are
\begin{figure}[H] \centering
\includegraphics[width=0.4\linewidth]{vo1f.pdf}
\caption{Phase of the gain in radians (Simulation)}
\label{fig:vov}
\end{figure}

\begin{figure}[H] \centering
\includegraphics[width=0.4\linewidth]{phase.eps}
\caption{Phase of the gain in radians (Theoretical Anlysis)}
\label{fig:vow}
\end{figure}

From the ngspice simulation OPAMP model, this component has two internal capacitors, (2 poles). This causes a sudden shift in the phase of $\frac{\pi}{2}$ per pole, so it rises sharply $\pi$ radians. This explains the difference between this plot and the one from the theoretical analysis.

\subsubsection{Central Frequency}
The maximum gain magnitude from the simultation is 35.65074dB.
We calculated the maximum gain-3dB = 32.65074 dB, to get the cutoff frequencies, and we got, from the simulation:\\
Lower Cutoff Frequency:408.3079 Hz\\
Upper Cutoff Frequency:2500.013 Hz\\
This gives a Central Frequency of 1010.3341 Hz.\\
The gain magnitude for the central frequency 1000Hz is 35.65074dB, which corresponds to 60.44711 (in linear units).
We also calculated the cutoff frequencies from the octave analysis, both using the formulas:
\begin{equation}
Low cutoff frequency=\frac{1}{2 \pi R_1 C_1},
\end{equation}
and
\begin{equation}
Upper cutoff frequency=\frac{1}{2 \pi R_2 C_2},
\end{equation}
having gotten the results 723.43Hz for the low frequency and 1446.9Hz for the upper, with a central frequency of 1023.1 Hz.\\
However, when we calculated the cutoff frequencies by intesecting  gain-3dB (gain= 35.657 dB) with the  gain plot (as we did in the simulation), we got the following results:\\
Lower Cutoff Frequency:406.95 Hz\\
Upper Cutoff Frequency:2572.1 Hz\\
This gives a Central Frequency of 1023.1 Hz.\\
These cutoff frequencies are much closer to the ones obtained by the simulation. However, the central frequency is not changed in both methods of theoretical analysis.


\subsection{Input and Output Impedances}
We want, ideally, to have a very large input impedance and a small output one.\\
From the simulation, for the central frequency of 1000Hz, we get the following results:\\
Input impedance: 999.9944 -723.526i $\Omega$\\
Output impedance:341.3639-233.092i $\Omega$\\
\\

We also calcuated, for this same frequency, the impedances from the theoretical analysis:
Input impedance:1000.00 -  723.43i $Omega$\\
Output impedance:338.37 - 233.86i $Omega$\\
\\
As we can see, both impedances are very close.\\

\subsection{Merit}
The merit of the circuit is given by
\begin{equation}
  M=\frac{1}{cost(voltage gain deviation+ central frequency deviation+10^{-6})}
\end{equation}
So to get the merit for our simulation,\\
Our central frequency deviation is 1010.3341-1000=10.3341Hz\\
Our gain deviation, from the wanted $40dB=10^{(40/20)}=100$ is 100-60.44711=39.55289.\\
The cost of the components of the circuit (capacitors and resistors aside from the OPAMP) is 0.22*2+10*3+1*3+100*3=333.44 M.U. The cost for the OPAMP is 13323.29204 M.U.\\
The total cost is then 13656.73204 M.U.\\
The merit then becomes $1.468\times 10^{-6}$.
