\section{Conclusions}
\label{sec:conc}
Our results were pretty consistent between the two analysis; the central frequencies were quite close (only around a 10Hz difference), the gain was exactly the same to the third decimal place and the plots of the frequency response were extremely similar, both displaying the same behaviour. \\
The subtle changes in the results can be explained by two facts. First of all in the theoretical model we consideral an ideal OPAMP so the gain is slightly larger in this analysis. Secondly, there are non-linear elements in the circuit (diodes, transistors, capacitors), these introduce some unexpected behaviour.\\
The differences in the phase plots are explained because in the ngspice simulation the OPAMP model has two internal capacitors, (2 poles). This causes a sudden shift in the phase of $\frac{\pi}{2}$ per pole, so it rises sharply $\pi$ radians.\\
The impedances again, are very consistent. The input impedance is large, as desired. The output impedance is a bit larger than expected, but it is still smaller and in another order of magnitude of the input impedance.\\
Finally, the merit calculated is quite low. This is a result of the high cost (especially taking into consideration the cost of the OPAMP). The gain deviation is also considerable, the central frequency deviation being the most successful.
