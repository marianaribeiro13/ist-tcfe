\section{Introduction}
\label{sec:intro}
The purpose of this work is to implement an active BandPass Filter (BPF) using one 741 OPAMP, capacitors and resistors, using a theoretical analysis and a simulation. The expected specifications of BandPass filter are a central frequency of 1kHz and a gain at central frequency of 40dB.\\
The circuit we constructed consisted of a High Pass filter stage, an amplification stage and a Low Pass filter stage. The High Pass filter stage has an AC voltage source $V_{in}$ of maximum voltage 10mV, a capacitor $C_1$ with capacitance 220nF and one resistor $R_1$ with 1k$\Omega$. The amplification stage has one 741 OPAMP, three resistors with resistance of 10k$\Omega$ in parallel creating an equivalent resistor $R_3$ and three resistors with resistance of 100k$\Omega$ in series creating an equivalent resistor $R_4$. The Low Pass filter stage has two resistors in parallel of 1k$\Omega$ each, resulting in equivalent resistor $R_2$ and a capacitor $C_2$ with capacitance 220nF. The circuit used can be seen in figure~\ref{fig:circngspice}.
\begin{figure}[H] \centering
\includegraphics[width=0.8\linewidth]{ngspice.pdf}
\caption{Circuit}
\label{fig:circngspice}
\end{figure} 

This circuit was used in both simulation and theoretical analysis, the only difference between them was the model used for the OPAMP. In the theoretical analysis we assumed an ideal OPAMP and in the simulation we used the 741 OPAMP model. In section~\ref{sec:res} we compared the frequency response (gain and phase), the central frequency achieved and the input and output impedances. 
