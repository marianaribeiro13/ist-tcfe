\section{Simulation}
\label{sec:simulation}
To make this simulation, we listed all components that make up the circuit. However, when declaring a current controlled voltage source ($V_c$, dependent on $I_c$), we need to find a way to access this current. For this, we need a voltage source where there isn't one which would sense the current $I_c$. Therefore, we created a fictitious one. This new source has voltage 0V and its sole purpose is to sense $I_c$, so it is in series with both resistors 6 and 7, between them (so their current should be the same). This makes node 6 connect $R_6$ to the positive terminal of this new source and create a new node that connects its negative terminal to $R_7$ (node 7). This makes the previous node 7 (that connects $R_7$ to $I_d$ and $V_c$) be called node 8.
In Table ~\ref{tab:op}, we can see the operating point analysis results from the \textit{ngspice} simulation. g1[i] corresponds to the current $I_b$; i1 is $I_d$ and the voltages correspond to each node. As we can see the values obtained through this analysis are exactly the same as the ones derived from theoretical analysis.
\begin{table}[H]
  \centering
  \begin{tabular}{|l|r|}
    \hline    
    {\bf Name} & {\bf Value [A or V]} \\ \hline
    @g1[i] & -2.44673e-04\\ \hline
@r1[i] & -2.33449e-04\\ \hline
@r2[i] & -2.44673e-04\\ \hline
@r3[i] & -1.12237e-05\\ \hline
@r4[i] & 1.225637e-03\\ \hline
@r5[i] & 2.446726e-04\\ \hline
@r6[i] & -9.92188e-04\\ \hline
@r7[i] & -9.92188e-04\\ \hline
v(1) & 5.233347e+00\\ \hline
v(2) & 4.993818e+00\\ \hline
v(3) & 4.495481e+00\\ \hline
v(5) & 5.028594e+00\\ \hline
v(6) & 5.796288e+00\\ \hline
v(7) & -2.00401e+00\\ \hline
v(8) & -3.03427e+00\\ \hline
v(9) & -2.00401e+00\\ \hline

  \end{tabular}
  \caption{Operating point. A variable preceded by @ is of type {\em current}
    and expressed in Ampere; other variables are of type {\it voltage} and expressed in
    Volt.}
  \label{tab:op}
\end{table}
