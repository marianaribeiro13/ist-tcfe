\section{Theoretical Analysis}
\label{sec:theo}
To make this analysis, we used the voltage obtained in equation ~\eqref{eq: vs}, as a source to mimic the voltage after passing through the transformer. 
\subsection{Envelope Detector}
To make the envelope detector, we used a full-wave bridge rectifier, where the diodes were approximated by voltage sources of voltage $V_{ON}=0.65$. The current would only pass through to the resistor and capacitor if $|V_S| \geq 2V_{ON}$, since there are two diodes polarized in each direction (if $V_S\geq 2V_{ON}$, it passes throught the forward-biased resistors and if $V_S \leq -2V_{ON}$, it passes through the backward biased ones). The rectified voltage through the resistor (and capacitor), $v_B$ and
\begin{equation}
    \begin{split}
{v_B (t)} = \left\{\begin{array}{lll} \ v_S(t) -2V_{ON},  \quad \ \ \ \ \  if \ \  \ v_S(t) \geq 2V_{ON} \\ 
 \ -v_S(t) -2V_{ON} , \quad \ \  if \ \ \ v_S(t) \leq -2V_{ON}\\
 \ 0,\quad \ \ \ \ \ \ \ \ \ \ \ \ \ \ \ \ \ \ \ \  if \ \  \ |v_S(t)|\textless 2V_{ON}\\
 \end{array} \right.
  \end{split}
\end{equation}
 We chose the resistor to have resistance $R=3k\Omega$ and the capacitor to have capacitance $C=3\mu F$ as well.\\
 To calculate the time where the diodes turn off, we used
 \begin{equation}
     t_{OFF}=\frac{1}{\omega}\arctan\left(\frac{1}{\omega RC}\right)
 \end{equation}
 During the time where the diodes are off $V_S \textless 2V_{ON}$, the capacitor discharges through the resistor and the voltage is
 \begin{equation}
     v_E(t)=(Acos(\omega t_{OFF})-2V_{ON})e^{-\frac{t-t_{OFF}}{RC}}
 \end{equation}
 The output voltage after passing through this envelope, $v_O$, becomes
 \begin{equation}
    \begin{split}
{v_O (t)} = \left\{\begin{array}{lll} \ v_B(t),  \quad \ \ \ \ \  if \ \  \ t \textless t_{OFF} \\ 
 \ v_E(t) , \quad \ \ \ \ \  if \   \ \ t \textgreater t_{OFF}\ \ and \ \ v_E(t)\textgreater v_B(t)\\
 \ v_B(t),\quad \ \ \  \ \  if \ \  \ t \textgreater t_{OFF} \ \ and v_E(t) \textless v_B(t)\\
 \end{array} \right.
  \end{split}
\end{equation}
This way the oscillations are reduced.
\subsection{Voltage Regulator}
To make the voltage regulator, the diodes were approximated using resistors. We used 18 diodes to be able to compare the results with the simulation. To calculate each resistance, we used 
\begin{equation}
 r_d=\frac{V_T}{I_se^{\frac{V_D}{\eta V_T}}}
\end{equation}
where $V_D=12/18 V$, ie the voltage divided by all the diodes. $I_S$ is the reverse saturated current of the diodes and we used $I_S=1 pA$, and we used the same $\eta$ as in the simulation: 1. $V_T$ is the thermal voltage, used 25mV, the value for 25\textdegree C. The resistor in Figure ~\ref{fig:oc2}, $R_D$ corresponds to $18r_d$, since they are connected in series and the input voltage is the one resulting from the envelope detector $v_O$.\\
The output voltage from this circuit is 
\begin{equation}
    v_o(t)=\frac{R_D}{R+R_D}v_O(t)
\end{equation}
where R is the resistor from the envelope detector.
This is only the AC part of the output.
The total output of this converter then is
\begin{equation}
    v_{OUT}(t)=12+v_o(t)
\end{equation}