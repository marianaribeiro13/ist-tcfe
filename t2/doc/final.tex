\subsection{Final Solution}
The final solution for $V_6(t)$ incorporates the natural and forced solution:
\begin{equation}
V_6(t) = V_{6n}(t) + V_{6f}(t)
\end{equation}
Before $V_6(t)$ is a constant (and so is $V_s(t)$), but in the instant t = 0, $V_6(0)=V_x$ and $V_s(0) = 0$. Then, for t > 0, $V_s(t) = sin(2\pi ft)$ and $V_6(t)$, from de sections ~\ref{section:n} and ~\ref{}, is:
\begin{equation}
V_6(t) = R(V_6)\times sin(wt)+Im(V_6)\times cos(wt)+Aexp(-\frac{t}{\tau})
\end{equation}
Finally, using \textit{Octave}, we plotted both $V_s(t)$ and $V_6(t)$ in the same figure in the interval [-5, 20] ms.